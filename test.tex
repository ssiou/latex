\documentclass{article}
\usepackage{amsmath} % For advanced math typesetting

\title{Sample LaTeX Document}
\author{Foo Bar}
\date{\today}

\begin{document}

\maketitle

\section{Introduction}
This is a sample document to demonstrate the use of LaTeX for writing integrals and equations.

\section{Integrals}
Here are some examples of integrals:

Inline integral: \( \int_{0}^{\infty} x^2 \, dx \)

Displayed integral:
\[
\int_{0}^{1} x^2 \, dx
\]

Double integral:
\[
\iint_{D}  \frac{1}{(x^2 + y^2)} \, dx \, dy
\]

Double integral 2 ranges:
\[
\int_{0}^{\infty} \int_{0}^{1}  \frac{1}{(x^2 + y^2)} \, dx \, dy
\]


Triple integral:
\[
\iiint_{V} (x^2 + y^2 + z^2) \, dx \, dy \, dz
\]

Triple integral 3 ranges:
\[
	\int_{0}^{1} \int_{0}^{\infty} \int_{-1}^{1} (x^2 + y^2 + z^2) \, dx \, dy \, dz
\]


\section{Math Equations}
Here are some examples of math equations:

Pythagorean theorem:
\[
a^2 + b^2 = c^2
\]

Euler's formula:
\[
e^{i\pi} + 1 = 0
\]

Quadratic formula:
\[
x = \frac{-b \pm \sqrt{b^2 - 4ac}}{2a}
\]

\section{Limits and Sums}
Here are some examples of limits and sums:

Limit:
\[
\lim_{x \to \infty} \frac{1}{x} = 0
\]

Sum:
\[
\sum_{n=1}^{\infty} \frac{1}{n^2} = \frac{\pi^2}{6}
\]

Product:
\[
\prod_{i=1}^{n} i = n!
\]

\section{Trigonometric Functions}

Here are some common trigonometric functions:

\[
\sin(x), \quad \cos(x), \quad \tan(x)
\]

Inverse trigonometric functions:

\[
\sin^{-1}(x), \quad \cos^{-1}(x), \quad \tan^{-1}(x)
\]

Hyperbolic trigonometric functions:

\[
\sinh(x), \quad \cosh(x), \quad \tanh(x)
\]

\section{Examples}

Here are some examples of trigonometric identities:

\[
\sin^2(x) + \cos^2(x) = 1
\]

\[
\tan(x) = \frac{\sin(x)}{\cos(x)}
\]

\[
\sin(2x) = 2 \sin(x) \cos(x)
\]

\section{Integrals Involving Trigonometric Functions}

Here are some integrals involving trigonometric functions:

\[
\int \sin(x) \, dx = -\cos(x) + C
\]

\[
\int \cos(x) \, dx = \sin(x) + C
\]

\[
\int \sec^2(x) \, dx = \tan(x) + C
\]
This is the theta symbol in inline math: $\theta$.

Here it is in a displayed equation:
\[
\theta = \frac{\pi}{4}
\]

\end{document}
